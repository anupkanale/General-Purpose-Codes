\documentclass[11pt, oneside]{article}   	% use "amsart" instead of "article" for AMSLaTeX format
\usepackage[margin=1in]{geometry}                		% See geometry.pdf to learn the layout options. There are lots.
\geometry{letterpaper}                   		% ... or a4paper or a5paper or ... 
%\geometry{landscape}                		% Activate for rotated page geometry
%\usepackage[parfill]{parskip}    		% Activate to begin paragraphs with an empty line rather than an indent
\usepackage{graphicx}				% Use pdf, png, jpg, or eps§ with pdflatex; use eps in DVI mode
								% TeX will automatically convert eps --> pdf in pdflatex		
\usepackage{amssymb}
\usepackage{undertilde}

\usepackage[T1]{fontenc}
\usepackage{mathtools}  % loads »amsmath«
\usepackage{physics}

\setlength{\parskip}{0.5em}
%SetFonts

%SetFonts
\newcommand\Rey{\mbox{\textit{Re}}}

\title{\vspace{-6ex}\Large METHOD OF LEAST SQUARES FOR A LINE}
\author{\vspace{-6ex}Anup V Kanale}
\date{\vspace{-3ex}\today}							% Activate to display a given date or no date

\begin{document}
\maketitle
\section*{Problem Statement}
For a velocity flow field $\boldsymbol{u} = u\boldsymbol{\hat{i}} + v \boldsymbol{\hat{j}} $, define a stream function such that
\begin{equation}
v = -\pdv{\psi}{x} \quad  u = \pdv{\psi}{y}
 \end{equation}
Taking curl of the Stokes flow equations, we get
\begin{equation}
 \nabla^2 (\nabla \times \boldsymbol{u}) = \nabla \times \nabla p = 0
\end{equation}
Writing in terms of the Stream function, we see that the equation is bi-harmonic
\begin{equation}
\nabla^4 \psi=0
\end{equation}
The streamlines for a Stokeslet in infinite domain are as shown in the figure below.
\begin{figure} [!htbp]
\centering
\includegraphics[scale=0.5]{StokesletStreamlines}
\caption{Flow past a Stokeslet}
\end{figure}
\end{document}